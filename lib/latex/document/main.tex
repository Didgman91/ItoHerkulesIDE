% !TeX spellcheck = en_GB

%%%%%%%%%%%%%%%%%%%%%%%%%%%%%%%%%%%%%%%%%
% Article Notes
% LaTeX Template
% Version 1.0 (1/10/15)
%
% This template has been downloaded from:
% http://www.LaTeXTemplates.com
%
% Authors:
% Vel (vel@latextemplates.com)
% Christopher Eliot (christopher.eliot@hofstra.edu)
% Anthony Dardis (anthony.dardis@hofstra.edu)
%
% License:
% CC BY-NC-SA 3.0 (http://creativecommons.org/licenses/by-nc-sa/3.0/)
%
%%%%%%%%%%%%%%%%%%%%%%%%%%%%%%%%%%%%%%%%%

%----------------------------------------------------------------------------------------
%	PACKAGES AND OTHER DOCUMENT CONFIGURATIONS
%----------------------------------------------------------------------------------------

\documentclass[
10pt, % Default font size is 10pt, can alternatively be 11pt or 12pt
a4paper, % Alternatively letterpaper for US letter
%twocolumn, % Alternatively onecolumn
onecolumn
%landscape % Alternatively portrait
portrait
]{article}

%!TEX root = article_notes.tex

%%%%%%%%%%%%%%%%%%%%%%%%%%%%%%%%%%%%%%%%%
% Article Notes
% Structure Specification File
% Version 1.0 (1/10/15)
%
% This file has been downloaded from:
% http://www.LaTeXTemplates.com
%
% Authors:
% Vel (vel@latextemplates.com)
% Christopher Eliot (christopher.eliot@hofstra.edu)
% Anthony Dardis (anthony.dardis@hofstra.edu)
%
% License:
% CC BY-NC-SA 3.0 (http://creativecommons.org/licenses/by-nc-sa/3.0/)
%
%%%%%%%%%%%%%%%%%%%%%%%%%%%%%%%%%%%%%%%%%

%----------------------------------------------------------------------------------------
%	REQUIRED PACKAGES
%----------------------------------------------------------------------------------------

\usepackage[includeheadfoot,columnsep=2cm, left=1in, right=1in, top=.5in, bottom=.5in]{geometry} % Margins

\usepackage[T1]{fontenc} % For international characters
\usepackage{XCharter} % XCharter as the main font

%\usepackage{natbib} % Use natbib to manage the reference
%\bibliographystyle{apalike} % Citation style
\usepackage[numbers,square]{natbib} % Use natbib to manage the reference
\bibliographystyle{alphadin} % Citation style

\usepackage[english]{babel} % Use english by default

%----------------------------------------------------------------------------------------
%	ADDITIONAL REQUIRED PACKAGES
%----------------------------------------------------------------------------------------

\usepackage{hyperref}		% hyperlinks

\usepackage{siunitx}		% SI units
\sisetup{detect-all}

\usepackage{attachfile}

\usepackage{graphicx}
\usepackage{import}
\usepackage{placeins}

\usepackage{amssymb}		% checkmark

\usepackage{xcolor}
\usepackage{listings}
\lstset{tabsize=2, breaklines=true, breakatwhitespace=true, frame=single}
\lstset{basicstyle=\footnotesize\ttfamily}

%----------------------------------------------------------------------------------------
%	CUSTOM COMMANDS
%----------------------------------------------------------------------------------------

\newcommand{\articletitle}[1]{\renewcommand{\articletitle}{#1}} % Define a command for storing the article title
\newcommand{\articlecitation}[1]{\renewcommand{\articlecitation}{#1}} % Define a command for storing the article citation
\newcommand{\doctitle}{\articlecitation\ --- \articletitle} % Define a command to store the article information as it will appear in the title and header
%\newcommand{\doctitle}{\articletitle} % Define a command to store the article information as it will appear in the title and header

\newcommand{\datenotesstarted}[1]{\renewcommand{\datenotesstarted}{#1}} % Define a command to store the date when notes were first made
\newcommand{\docdate}[1]{\renewcommand{\docdate}{#1}} % Define a command to store the date line in the title

\newcommand{\docauthor}[1]{\renewcommand{\docauthor}{#1}} % Define a command for storing the article notes author

% Define a command for the structure of the document title
\newcommand{\printtitle}{
\begin{center}
\textbf{\Large{\doctitle}}

\docdate

\docauthor
\end{center}
}

%----------------------------------------------------------------------------------------
%	STRUCTURE MODIFICATIONS
%----------------------------------------------------------------------------------------

\setlength{\parindent}{0pt} % no indent at new paragraph

\setlength{\parskip}{3pt} % Slightly increase spacing between paragraphs

% Uncomment to center section titles
%\usepackage{sectsty}
%\sectionfont{\centering}

% Uncomment for Roman numerals for section numbers
%\renewcommand\thesection{\Roman{section}}
 % Input the file specifying the document layout and structure

%----------------------------------------------------------------------------------------
%	ARTICLE INFORMATION
%----------------------------------------------------------------------------------------

\articletitle{KW0} % The title of the article
\articlecitation{Notes} % The BibTeX citation key from your bibliography

\datenotesstarted{November 13, 2018} % The date when these notes were first made
\docdate{\datenotesstarted; rev. \today} % The date when the notes were lasted updated (automatically the current date)

\docauthor{Max Daiber-Huppert} % Your name

%----------------------------------------------------------------------------------------

\begin{document}

\pagestyle{myheadings} % Use custom headers
\markright{\doctitle} % Place the article information into the header

%----------------------------------------------------------------------------------------
%	PRINT ARTICLE INFORMATION
%----------------------------------------------------------------------------------------

\thispagestyle{plain} % Plain formatting on the first page

\printtitle % Print the title

\FloatBarrier
\section{Plot}

\begin{tikzpicture}[y=.2cm, x=.7cm,font=\sffamily]
%axis
\draw (0,0) -- coordinate (x axis mid) (10,0);
\draw (0,0) -- coordinate (y axis mid) (0,30);
%ticks
\foreach \x in {0,...,10}
\draw (\x,1pt) -- (\x,-3pt)
node[anchor=north] {\x};
\foreach \y in {0,5,...,30}
\draw (1pt,\y) -- (-3pt,\y) 
node[anchor=east] {\y}; 
%labels      
\node[below=0.8cm] at (x axis mid) {MOPS};
\node[rotate=90, above=0.8cm] at (y axis mid) {Power [mW]};

%plots
\draw plot[mark=*, mark options={fill=white}] 
file {csv/div_soft.data};
\draw plot[mark=triangle*, mark options={fill=white} ] 
file {csv/div_ciu.data};
\draw plot[mark=square*, mark options={fill=white}]
file {csv/div_ciu_oscar.data};
\draw plot[mark=square*]
file {csv/div_ciu_oscar_extrapolated.data};  

%legend
\begin{scope}[shift={(4,4)}] 
\draw (0,0) -- 
plot[mark=*, mark options={fill=white}] (0.25,0) -- (0.5,0) 
node[right]{soft};
\draw[yshift=\baselineskip] (0,0) -- 
plot[mark=triangle*, mark options={fill=white}] (0.25,0) -- (0.5,0)
node[right]{ciu};
\draw[yshift=2\baselineskip] (0,0) -- 
plot[mark=square*, mark options={fill=white}] (0.25,0) -- (0.5,0)
node[right]{ciu + oscar};
\draw[yshift=3\baselineskip] (0,0) -- 
plot[mark=square*, mark options={fill=black}] (0.25,0) -- (0.5,0)
node[right]{ciu + oscar extrapolated};
\end{scope}
\end{tikzpicture}


\FloatBarrier
\subsection{Results}

%----------------------------------------------------------------------------------------
%	ATTACHMENT
%----------------------------------------------------------------------------------------

%\section{Attachment}
%Deep Speckle Correlation: \attachfile[icon=Paperclip]{Attachment/Deep_Speckle_correlation.pdf}

%----------------------------------------------------------------------------------------
%	BIBLIOGRAPHY
%----------------------------------------------------------------------------------------

\renewcommand{\refname}{Reference} % Change the default bibliography title

\bibliography{sample} % Input your bibliography file

%----------------------------------------------------------------------------------------

\end{document}
